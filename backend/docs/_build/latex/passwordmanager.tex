%% Generated by Sphinx.
\def\sphinxdocclass{report}
\documentclass[letterpaper,10pt,polish]{sphinxmanual}
\ifdefined\pdfpxdimen
   \let\sphinxpxdimen\pdfpxdimen\else\newdimen\sphinxpxdimen
\fi \sphinxpxdimen=.75bp\relax
\ifdefined\pdfimageresolution
    \pdfimageresolution= \numexpr \dimexpr1in\relax/\sphinxpxdimen\relax
\fi
%% let collapsible pdf bookmarks panel have high depth per default
\PassOptionsToPackage{bookmarksdepth=5}{hyperref}

\PassOptionsToPackage{booktabs}{sphinx}
\PassOptionsToPackage{colorrows}{sphinx}

\PassOptionsToPackage{warn}{textcomp}
\usepackage[utf8]{inputenc}
\ifdefined\DeclareUnicodeCharacter
% support both utf8 and utf8x syntaxes
  \ifdefined\DeclareUnicodeCharacterAsOptional
    \def\sphinxDUC#1{\DeclareUnicodeCharacter{"#1}}
  \else
    \let\sphinxDUC\DeclareUnicodeCharacter
  \fi
  \sphinxDUC{00A0}{\nobreakspace}
  \sphinxDUC{2500}{\sphinxunichar{2500}}
  \sphinxDUC{2502}{\sphinxunichar{2502}}
  \sphinxDUC{2514}{\sphinxunichar{2514}}
  \sphinxDUC{251C}{\sphinxunichar{251C}}
  \sphinxDUC{2572}{\textbackslash}
\fi
\usepackage{cmap}
\usepackage[T1]{fontenc}
\usepackage{amsmath,amssymb,amstext}
\usepackage{babel}



\usepackage{tgtermes}
\usepackage{tgheros}
\renewcommand{\ttdefault}{txtt}



\usepackage[Sonny]{fncychap}
\ChNameVar{\Large\normalfont\sffamily}
\ChTitleVar{\Large\normalfont\sffamily}
\usepackage{sphinx}

\fvset{fontsize=auto}
\usepackage{geometry}


% Include hyperref last.
\usepackage{hyperref}
% Fix anchor placement for figures with captions.
\usepackage{hypcap}% it must be loaded after hyperref.
% Set up styles of URL: it should be placed after hyperref.
\urlstyle{same}

\addto\captionspolish{\renewcommand{\contentsname}{Contents:}}

\usepackage{sphinxmessages}
\setcounter{tocdepth}{1}



\title{Password Manager}
\date{24 lis 2023}
\release{1.0.0}
\author{Patryk Jaworski, Jakub Jach}
\newcommand{\sphinxlogo}{\vbox{}}
\renewcommand{\releasename}{Wydanie}
\makeindex
\begin{document}

\ifdefined\shorthandoff
  \ifnum\catcode`\=\string=\active\shorthandoff{=}\fi
  \ifnum\catcode`\"=\active\shorthandoff{"}\fi
\fi

\pagestyle{empty}
\sphinxmaketitle
\pagestyle{plain}
\sphinxtableofcontents
\pagestyle{normal}
\phantomsection\label{\detokenize{index::doc}}


\sphinxstepscope


\chapter{backend}
\label{\detokenize{modules:backend}}\label{\detokenize{modules::doc}}
\sphinxstepscope


\section{backend package}
\label{\detokenize{backend:backend-package}}\label{\detokenize{backend::doc}}

\subsection{Submodules}
\label{\detokenize{backend:submodules}}

\subsection{backend.asgi module}
\label{\detokenize{backend:module-backend.asgi}}\label{\detokenize{backend:backend-asgi-module}}\index{module@\spxentry{module}!backend.asgi@\spxentry{backend.asgi}}\index{backend.asgi@\spxentry{backend.asgi}!module@\spxentry{module}}
\sphinxAtStartPar
ASGI config for projectBAI project.

\sphinxAtStartPar
It exposes the ASGI callable as a module\sphinxhyphen{}level variable named \sphinxcode{\sphinxupquote{application}}.

\sphinxAtStartPar
For more information on this file, see
\sphinxurl{https://docs.djangoproject.com/en/4.2/howto/deployment/asgi/}


\subsection{backend.settings module}
\label{\detokenize{backend:module-backend.settings}}\label{\detokenize{backend:backend-settings-module}}\index{module@\spxentry{module}!backend.settings@\spxentry{backend.settings}}\index{backend.settings@\spxentry{backend.settings}!module@\spxentry{module}}
\sphinxAtStartPar
Django settings for projectBAI project.

\sphinxAtStartPar
Generated by «django\sphinxhyphen{}admin startproject» using Django 4.2.4.

\sphinxAtStartPar
For more information on this file, see
\sphinxurl{https://docs.djangoproject.com/en/4.2/topics/settings/}

\sphinxAtStartPar
For the full list of settings and their values, see
\sphinxurl{https://docs.djangoproject.com/en/4.2/ref/settings/}


\subsection{backend.urls module}
\label{\detokenize{backend:module-backend.urls}}\label{\detokenize{backend:backend-urls-module}}\index{module@\spxentry{module}!backend.urls@\spxentry{backend.urls}}\index{backend.urls@\spxentry{backend.urls}!module@\spxentry{module}}
\sphinxAtStartPar
URL configuration for projectBAI project.
\begin{description}
\sphinxlineitem{The \sphinxtitleref{urlpatterns} list routes URLs to views. For more information please see:}
\sphinxAtStartPar
\sphinxurl{https://docs.djangoproject.com/en/4.2/topics/http/urls/}

\end{description}

\sphinxAtStartPar
Examples:
Function views
\begin{enumerate}
\sphinxsetlistlabels{\arabic}{enumi}{enumii}{}{.}%
\item {} 
\sphinxAtStartPar
Add an import:  from my\_app import views

\item {} 
\sphinxAtStartPar
Add a URL to urlpatterns:  path(«», views.home, name=»home»)

\end{enumerate}
\begin{description}
\sphinxlineitem{Class\sphinxhyphen{}based views}\begin{enumerate}
\sphinxsetlistlabels{\arabic}{enumi}{enumii}{}{.}%
\item {} 
\sphinxAtStartPar
Add an import:  from other\_app.views import Home

\item {} 
\sphinxAtStartPar
Add a URL to urlpatterns:  path(«», Home.as\_view(), name=»home»)

\end{enumerate}

\sphinxlineitem{Including another URLconf}\begin{enumerate}
\sphinxsetlistlabels{\arabic}{enumi}{enumii}{}{.}%
\item {} 
\sphinxAtStartPar
Import the include() function: from django.urls import include, path

\item {} 
\sphinxAtStartPar
Add a URL to urlpatterns:  path(«blog/», include(«blog.urls»))

\end{enumerate}

\end{description}


\subsection{backend.wsgi module}
\label{\detokenize{backend:module-backend.wsgi}}\label{\detokenize{backend:backend-wsgi-module}}\index{module@\spxentry{module}!backend.wsgi@\spxentry{backend.wsgi}}\index{backend.wsgi@\spxentry{backend.wsgi}!module@\spxentry{module}}
\sphinxAtStartPar
WSGI config for projectBAI project.

\sphinxAtStartPar
It exposes the WSGI callable as a module\sphinxhyphen{}level variable named \sphinxcode{\sphinxupquote{application}}.

\sphinxAtStartPar
For more information on this file, see
\sphinxurl{https://docs.djangoproject.com/en/4.2/howto/deployment/wsgi/}


\subsection{Module contents}
\label{\detokenize{backend:module-backend}}\label{\detokenize{backend:module-contents}}\index{module@\spxentry{module}!backend@\spxentry{backend}}\index{backend@\spxentry{backend}!module@\spxentry{module}}
\sphinxstepscope


\section{mainApp package}
\label{\detokenize{mainApp:mainapp-package}}\label{\detokenize{mainApp::doc}}

\subsection{Subpackages}
\label{\detokenize{mainApp:subpackages}}

\subsection{Submodules}
\label{\detokenize{mainApp:submodules}}

\subsection{mainApp.admin module}
\label{\detokenize{mainApp:module-mainApp.admin}}\label{\detokenize{mainApp:mainapp-admin-module}}\index{module@\spxentry{module}!mainApp.admin@\spxentry{mainApp.admin}}\index{mainApp.admin@\spxentry{mainApp.admin}!module@\spxentry{module}}

\subsection{mainApp.apps module}
\label{\detokenize{mainApp:module-mainApp.apps}}\label{\detokenize{mainApp:mainapp-apps-module}}\index{module@\spxentry{module}!mainApp.apps@\spxentry{mainApp.apps}}\index{mainApp.apps@\spxentry{mainApp.apps}!module@\spxentry{module}}\index{MainappConfig (klasa w module mainApp.apps)@\spxentry{MainappConfig}\spxextra{klasa w module mainApp.apps}}

\begin{fulllineitems}
\phantomsection\label{\detokenize{mainApp:mainApp.apps.MainappConfig}}
\pysigstartsignatures
\pysiglinewithargsret{\sphinxbfcode{\sphinxupquote{class\DUrole{w}{ }}}\sphinxcode{\sphinxupquote{mainApp.apps.}}\sphinxbfcode{\sphinxupquote{MainappConfig}}}{\sphinxparam{\DUrole{n}{app\_name}}\sphinxparamcomma \sphinxparam{\DUrole{n}{app\_module}}}{}
\pysigstopsignatures
\sphinxAtStartPar
Klasy bazowe: \sphinxcode{\sphinxupquote{AppConfig}}

\sphinxAtStartPar
Konfiguracja aplikacji Django o nazwie «mainApp».

\sphinxAtStartPar
Klasa \sphinxtitleref{MainappConfig} definiuje konfigurację głównej aplikacji Django o nazwie «mainApp».
Zawiera parametr \sphinxtitleref{default\_auto\_field}, który określa domyślne pole automatyczne dla modeli.
\begin{quote}\begin{description}
\sphinxlineitem{Zmienne}\begin{itemize}
\item {} 
\sphinxAtStartPar
\sphinxstyleliteralstrong{\sphinxupquote{default\_auto\_field}} (\sphinxstyleliteralemphasis{\sphinxupquote{str}}) \textendash{} Pole automatyczne używane jako domyślne dla modeli w aplikacji.

\item {} 
\sphinxAtStartPar
\sphinxstyleliteralstrong{\sphinxupquote{name}} (\sphinxstyleliteralemphasis{\sphinxupquote{str}}) \textendash{} Nazwa aplikacji Django.

\end{itemize}

\end{description}\end{quote}
\index{default\_auto\_field (mainApp.apps.MainappConfig atrybut)@\spxentry{default\_auto\_field}\spxextra{mainApp.apps.MainappConfig atrybut}}

\begin{fulllineitems}
\phantomsection\label{\detokenize{mainApp:mainApp.apps.MainappConfig.default_auto_field}}
\pysigstartsignatures
\pysigline{\sphinxbfcode{\sphinxupquote{default\_auto\_field}}\sphinxbfcode{\sphinxupquote{\DUrole{w}{ }\DUrole{p}{=}\DUrole{w}{ }\textquotesingle{}django.db.models.BigAutoField\textquotesingle{}}}}
\pysigstopsignatures
\end{fulllineitems}

\index{name (mainApp.apps.MainappConfig atrybut)@\spxentry{name}\spxextra{mainApp.apps.MainappConfig atrybut}}

\begin{fulllineitems}
\phantomsection\label{\detokenize{mainApp:mainApp.apps.MainappConfig.name}}
\pysigstartsignatures
\pysigline{\sphinxbfcode{\sphinxupquote{name}}\sphinxbfcode{\sphinxupquote{\DUrole{w}{ }\DUrole{p}{=}\DUrole{w}{ }\textquotesingle{}mainApp\textquotesingle{}}}}
\pysigstopsignatures
\end{fulllineitems}


\end{fulllineitems}



\subsection{mainApp.forms module}
\label{\detokenize{mainApp:module-mainApp.forms}}\label{\detokenize{mainApp:mainapp-forms-module}}\index{module@\spxentry{module}!mainApp.forms@\spxentry{mainApp.forms}}\index{mainApp.forms@\spxentry{mainApp.forms}!module@\spxentry{module}}\index{UserForm (klasa w module mainApp.forms)@\spxentry{UserForm}\spxextra{klasa w module mainApp.forms}}

\begin{fulllineitems}
\phantomsection\label{\detokenize{mainApp:mainApp.forms.UserForm}}
\pysigstartsignatures
\pysiglinewithargsret{\sphinxbfcode{\sphinxupquote{class\DUrole{w}{ }}}\sphinxcode{\sphinxupquote{mainApp.forms.}}\sphinxbfcode{\sphinxupquote{UserForm}}}{\sphinxparam{\DUrole{o}{*}\DUrole{n}{args}}\sphinxparamcomma \sphinxparam{\DUrole{o}{**}\DUrole{n}{kwargs}}}{}
\pysigstopsignatures
\sphinxAtStartPar
Klasy bazowe: \sphinxcode{\sphinxupquote{UserCreationForm}}

\sphinxAtStartPar
Formularz rejestracji użytkownika.

\sphinxAtStartPar
Klasa \sphinxtitleref{UserForm} dziedziczy po \sphinxtitleref{UserCreationForm} i dodaje pole email.
Używa modelu \sphinxtitleref{User} i zawiera funkcję \sphinxtitleref{save()}, która zapisuje użytkownika
do bazy danych.
\begin{quote}\begin{description}
\sphinxlineitem{Zmienne}
\sphinxAtStartPar
\sphinxstyleliteralstrong{\sphinxupquote{email}} (\sphinxstyleliteralemphasis{\sphinxupquote{forms.EmailField}}) \textendash{} Pole email użytkownika.

\end{description}\end{quote}
\index{UserForm.Meta (klasa w module mainApp.forms)@\spxentry{UserForm.Meta}\spxextra{klasa w module mainApp.forms}}

\begin{fulllineitems}
\phantomsection\label{\detokenize{mainApp:mainApp.forms.UserForm.Meta}}
\pysigstartsignatures
\pysigline{\sphinxbfcode{\sphinxupquote{class\DUrole{w}{ }}}\sphinxbfcode{\sphinxupquote{Meta}}}
\pysigstopsignatures
\sphinxAtStartPar
Klasy bazowe: \sphinxcode{\sphinxupquote{object}}
\index{fields (mainApp.forms.UserForm.Meta atrybut)@\spxentry{fields}\spxextra{mainApp.forms.UserForm.Meta atrybut}}

\begin{fulllineitems}
\phantomsection\label{\detokenize{mainApp:mainApp.forms.UserForm.Meta.fields}}
\pysigstartsignatures
\pysigline{\sphinxbfcode{\sphinxupquote{fields}}\sphinxbfcode{\sphinxupquote{\DUrole{w}{ }\DUrole{p}{=}\DUrole{w}{ }(\textquotesingle{}username\textquotesingle{}, \textquotesingle{}email\textquotesingle{}, \textquotesingle{}password1\textquotesingle{}, \textquotesingle{}password2\textquotesingle{})}}}
\pysigstopsignatures
\end{fulllineitems}

\index{model (mainApp.forms.UserForm.Meta atrybut)@\spxentry{model}\spxextra{mainApp.forms.UserForm.Meta atrybut}}

\begin{fulllineitems}
\phantomsection\label{\detokenize{mainApp:mainApp.forms.UserForm.Meta.model}}
\pysigstartsignatures
\pysigline{\sphinxbfcode{\sphinxupquote{model}}}
\pysigstopsignatures
\sphinxAtStartPar
alias of \sphinxcode{\sphinxupquote{User}}

\end{fulllineitems}


\end{fulllineitems}

\index{base\_fields (mainApp.forms.UserForm atrybut)@\spxentry{base\_fields}\spxextra{mainApp.forms.UserForm atrybut}}

\begin{fulllineitems}
\phantomsection\label{\detokenize{mainApp:mainApp.forms.UserForm.base_fields}}
\pysigstartsignatures
\pysigline{\sphinxbfcode{\sphinxupquote{base\_fields}}\sphinxbfcode{\sphinxupquote{\DUrole{w}{ }\DUrole{p}{=}\DUrole{w}{ }\{\textquotesingle{}email\textquotesingle{}: \textless{}django.forms.fields.EmailField object\textgreater{}, \textquotesingle{}password1\textquotesingle{}: \textless{}django.forms.fields.CharField object\textgreater{}, \textquotesingle{}password2\textquotesingle{}: \textless{}django.forms.fields.CharField object\textgreater{}, \textquotesingle{}username\textquotesingle{}: \textless{}django.forms.fields.CharField object\textgreater{}\}}}}
\pysigstopsignatures
\end{fulllineitems}

\index{declared\_fields (mainApp.forms.UserForm atrybut)@\spxentry{declared\_fields}\spxextra{mainApp.forms.UserForm atrybut}}

\begin{fulllineitems}
\phantomsection\label{\detokenize{mainApp:mainApp.forms.UserForm.declared_fields}}
\pysigstartsignatures
\pysigline{\sphinxbfcode{\sphinxupquote{declared\_fields}}\sphinxbfcode{\sphinxupquote{\DUrole{w}{ }\DUrole{p}{=}\DUrole{w}{ }\{\textquotesingle{}email\textquotesingle{}: \textless{}django.forms.fields.EmailField object\textgreater{}, \textquotesingle{}password1\textquotesingle{}: \textless{}django.forms.fields.CharField object\textgreater{}, \textquotesingle{}password2\textquotesingle{}: \textless{}django.forms.fields.CharField object\textgreater{}\}}}}
\pysigstopsignatures
\end{fulllineitems}

\index{media (mainApp.forms.UserForm property)@\spxentry{media}\spxextra{mainApp.forms.UserForm property}}

\begin{fulllineitems}
\phantomsection\label{\detokenize{mainApp:mainApp.forms.UserForm.media}}
\pysigstartsignatures
\pysigline{\sphinxbfcode{\sphinxupquote{property\DUrole{w}{ }}}\sphinxbfcode{\sphinxupquote{media}}}
\pysigstopsignatures
\sphinxAtStartPar
Return all media required to render the widgets on this form.

\end{fulllineitems}

\index{save() (mainApp.forms.UserForm metoda)@\spxentry{save()}\spxextra{mainApp.forms.UserForm metoda}}

\begin{fulllineitems}
\phantomsection\label{\detokenize{mainApp:mainApp.forms.UserForm.save}}
\pysigstartsignatures
\pysiglinewithargsret{\sphinxbfcode{\sphinxupquote{save}}}{\sphinxparam{\DUrole{n}{commit}\DUrole{o}{=}\DUrole{default_value}{True}}}{}
\pysigstopsignatures
\sphinxAtStartPar
Zapisuje użytkownika do bazy danych.

\sphinxAtStartPar
Funkcja \sphinxtitleref{save()} zapisuje nowego użytkownika do bazy danych.
Jeśli \sphinxtitleref{commit} jest ustawione na \sphinxtitleref{True}, użytkownik zostanie zapisany
do bazy danych.
\begin{quote}\begin{description}
\sphinxlineitem{Parametry}
\sphinxAtStartPar
\sphinxstyleliteralstrong{\sphinxupquote{commit}} (\sphinxstyleliteralemphasis{\sphinxupquote{bool}}) \textendash{} Flaga wskazująca, czy zapisać użytkownika do bazy danych. Domyślnie True.

\sphinxlineitem{Zwraca}
\sphinxAtStartPar
Użytkownik zapisany w bazie danych.

\sphinxlineitem{Typ zwracany}
\sphinxAtStartPar
User

\end{description}\end{quote}

\end{fulllineitems}


\end{fulllineitems}



\subsection{mainApp.models module}
\label{\detokenize{mainApp:module-mainApp.models}}\label{\detokenize{mainApp:mainapp-models-module}}\index{module@\spxentry{module}!mainApp.models@\spxentry{mainApp.models}}\index{mainApp.models@\spxentry{mainApp.models}!module@\spxentry{module}}\index{CreditStorage (klasa w module mainApp.models)@\spxentry{CreditStorage}\spxextra{klasa w module mainApp.models}}

\begin{fulllineitems}
\phantomsection\label{\detokenize{mainApp:mainApp.models.CreditStorage}}
\pysigstartsignatures
\pysiglinewithargsret{\sphinxbfcode{\sphinxupquote{class\DUrole{w}{ }}}\sphinxcode{\sphinxupquote{mainApp.models.}}\sphinxbfcode{\sphinxupquote{CreditStorage}}}{\sphinxparam{\DUrole{o}{*}\DUrole{n}{args}}\sphinxparamcomma \sphinxparam{\DUrole{o}{**}\DUrole{n}{kwargs}}}{}
\pysigstopsignatures
\sphinxAtStartPar
Klasy bazowe: \sphinxcode{\sphinxupquote{Model}}

\sphinxAtStartPar
Przechowywanie danych związanego z zapisanymi serwisami.

\sphinxAtStartPar
Klasa \sphinxtitleref{CreditStorage} definiuje model przechowujący dane związane z serwisami.
Powiązany jest z modelem \sphinxtitleref{User} reprezentującym użytkownika.
\begin{quote}\begin{description}
\sphinxlineitem{Zmienne}\begin{itemize}
\item {} 
\sphinxAtStartPar
\sphinxstyleliteralstrong{\sphinxupquote{user}} (\sphinxstyleliteralemphasis{\sphinxupquote{User}}) \textendash{} Powiązanie z modelem użytkownika Django.

\item {} 
\sphinxAtStartPar
\sphinxstyleliteralstrong{\sphinxupquote{name}} (\sphinxstyleliteralemphasis{\sphinxupquote{str}}) \textendash{} Nazwa przechowywanych danych.

\item {} 
\sphinxAtStartPar
\sphinxstyleliteralstrong{\sphinxupquote{icon}} (\sphinxstyleliteralemphasis{\sphinxupquote{dict}}) \textendash{} Ikona powiązana z danymi serwisu (w formacie JSON).

\item {} 
\sphinxAtStartPar
\sphinxstyleliteralstrong{\sphinxupquote{username}} (\sphinxstyleliteralemphasis{\sphinxupquote{encrypt}}) \textendash{} Zaszyfrowane pole przechowujące nazwę użytkownika.

\item {} 
\sphinxAtStartPar
\sphinxstyleliteralstrong{\sphinxupquote{password}} (\sphinxstyleliteralemphasis{\sphinxupquote{encrypt}}) \textendash{} Zaszyfrowane pole przechowujące hasło użytkownika.

\end{itemize}

\end{description}\end{quote}
\index{CreditStorage.DoesNotExist@\spxentry{CreditStorage.DoesNotExist}}

\begin{fulllineitems}
\phantomsection\label{\detokenize{mainApp:mainApp.models.CreditStorage.DoesNotExist}}
\pysigstartsignatures
\pysigline{\sphinxbfcode{\sphinxupquote{exception\DUrole{w}{ }}}\sphinxbfcode{\sphinxupquote{DoesNotExist}}}
\pysigstopsignatures
\sphinxAtStartPar
Klasy bazowe: \sphinxcode{\sphinxupquote{ObjectDoesNotExist}}

\end{fulllineitems}

\index{CreditStorage.MultipleObjectsReturned@\spxentry{CreditStorage.MultipleObjectsReturned}}

\begin{fulllineitems}
\phantomsection\label{\detokenize{mainApp:mainApp.models.CreditStorage.MultipleObjectsReturned}}
\pysigstartsignatures
\pysigline{\sphinxbfcode{\sphinxupquote{exception\DUrole{w}{ }}}\sphinxbfcode{\sphinxupquote{MultipleObjectsReturned}}}
\pysigstopsignatures
\sphinxAtStartPar
Klasy bazowe: \sphinxcode{\sphinxupquote{MultipleObjectsReturned}}

\end{fulllineitems}

\index{icon (mainApp.models.CreditStorage atrybut)@\spxentry{icon}\spxextra{mainApp.models.CreditStorage atrybut}}

\begin{fulllineitems}
\phantomsection\label{\detokenize{mainApp:mainApp.models.CreditStorage.icon}}
\pysigstartsignatures
\pysigline{\sphinxbfcode{\sphinxupquote{icon}}}
\pysigstopsignatures
\sphinxAtStartPar
A wrapper for a deferred\sphinxhyphen{}loading field. When the value is read from this
object the first time, the query is executed.

\end{fulllineitems}

\index{id (mainApp.models.CreditStorage atrybut)@\spxentry{id}\spxextra{mainApp.models.CreditStorage atrybut}}

\begin{fulllineitems}
\phantomsection\label{\detokenize{mainApp:mainApp.models.CreditStorage.id}}
\pysigstartsignatures
\pysigline{\sphinxbfcode{\sphinxupquote{id}}}
\pysigstopsignatures
\sphinxAtStartPar
A wrapper for a deferred\sphinxhyphen{}loading field. When the value is read from this
object the first time, the query is executed.

\end{fulllineitems}

\index{name (mainApp.models.CreditStorage atrybut)@\spxentry{name}\spxextra{mainApp.models.CreditStorage atrybut}}

\begin{fulllineitems}
\phantomsection\label{\detokenize{mainApp:mainApp.models.CreditStorage.name}}
\pysigstartsignatures
\pysigline{\sphinxbfcode{\sphinxupquote{name}}}
\pysigstopsignatures
\sphinxAtStartPar
A wrapper for a deferred\sphinxhyphen{}loading field. When the value is read from this
object the first time, the query is executed.

\end{fulllineitems}

\index{objects (mainApp.models.CreditStorage atrybut)@\spxentry{objects}\spxextra{mainApp.models.CreditStorage atrybut}}

\begin{fulllineitems}
\phantomsection\label{\detokenize{mainApp:mainApp.models.CreditStorage.objects}}
\pysigstartsignatures
\pysigline{\sphinxbfcode{\sphinxupquote{objects}}\sphinxbfcode{\sphinxupquote{\DUrole{w}{ }\DUrole{p}{=}\DUrole{w}{ }\textless{}django.db.models.manager.Manager object\textgreater{}}}}
\pysigstopsignatures
\end{fulllineitems}

\index{password (mainApp.models.CreditStorage atrybut)@\spxentry{password}\spxextra{mainApp.models.CreditStorage atrybut}}

\begin{fulllineitems}
\phantomsection\label{\detokenize{mainApp:mainApp.models.CreditStorage.password}}
\pysigstartsignatures
\pysigline{\sphinxbfcode{\sphinxupquote{password}}}
\pysigstopsignatures
\sphinxAtStartPar
A wrapper for a deferred\sphinxhyphen{}loading field. When the value is read from this
object the first time, the query is executed.

\end{fulllineitems}

\index{user (mainApp.models.CreditStorage atrybut)@\spxentry{user}\spxextra{mainApp.models.CreditStorage atrybut}}

\begin{fulllineitems}
\phantomsection\label{\detokenize{mainApp:mainApp.models.CreditStorage.user}}
\pysigstartsignatures
\pysigline{\sphinxbfcode{\sphinxupquote{user}}}
\pysigstopsignatures
\sphinxAtStartPar
Accessor to the related object on the forward side of a many\sphinxhyphen{}to\sphinxhyphen{}one or
one\sphinxhyphen{}to\sphinxhyphen{}one (via ForwardOneToOneDescriptor subclass) relation.

\sphinxAtStartPar
In the example:

\begin{sphinxVerbatim}[commandchars=\\\{\}]
\PYG{k}{class} \PYG{n+nc}{Child}\PYG{p}{(}\PYG{n}{Model}\PYG{p}{)}\PYG{p}{:}
    \PYG{n}{parent} \PYG{o}{=} \PYG{n}{ForeignKey}\PYG{p}{(}\PYG{n}{Parent}\PYG{p}{,} \PYG{n}{related\PYGZus{}name}\PYG{o}{=}\PYG{l+s+s1}{\PYGZsq{}}\PYG{l+s+s1}{children}\PYG{l+s+s1}{\PYGZsq{}}\PYG{p}{)}
\end{sphinxVerbatim}

\sphinxAtStartPar
\sphinxcode{\sphinxupquote{Child.parent}} is a \sphinxcode{\sphinxupquote{ForwardManyToOneDescriptor}} instance.

\end{fulllineitems}

\index{user\_id (mainApp.models.CreditStorage atrybut)@\spxentry{user\_id}\spxextra{mainApp.models.CreditStorage atrybut}}

\begin{fulllineitems}
\phantomsection\label{\detokenize{mainApp:mainApp.models.CreditStorage.user_id}}
\pysigstartsignatures
\pysigline{\sphinxbfcode{\sphinxupquote{user\_id}}}
\pysigstopsignatures
\end{fulllineitems}

\index{username (mainApp.models.CreditStorage atrybut)@\spxentry{username}\spxextra{mainApp.models.CreditStorage atrybut}}

\begin{fulllineitems}
\phantomsection\label{\detokenize{mainApp:mainApp.models.CreditStorage.username}}
\pysigstartsignatures
\pysigline{\sphinxbfcode{\sphinxupquote{username}}}
\pysigstopsignatures
\sphinxAtStartPar
A wrapper for a deferred\sphinxhyphen{}loading field. When the value is read from this
object the first time, the query is executed.

\end{fulllineitems}


\end{fulllineitems}

\index{UserProfile (klasa w module mainApp.models)@\spxentry{UserProfile}\spxextra{klasa w module mainApp.models}}

\begin{fulllineitems}
\phantomsection\label{\detokenize{mainApp:mainApp.models.UserProfile}}
\pysigstartsignatures
\pysiglinewithargsret{\sphinxbfcode{\sphinxupquote{class\DUrole{w}{ }}}\sphinxcode{\sphinxupquote{mainApp.models.}}\sphinxbfcode{\sphinxupquote{UserProfile}}}{\sphinxparam{\DUrole{o}{*}\DUrole{n}{args}}\sphinxparamcomma \sphinxparam{\DUrole{o}{**}\DUrole{n}{kwargs}}}{}
\pysigstopsignatures
\sphinxAtStartPar
Klasy bazowe: \sphinxcode{\sphinxupquote{Model}}

\sphinxAtStartPar
Profil użytkownika aplikacji.

\sphinxAtStartPar
Klasa \sphinxtitleref{UserProfile} definiuje profil użytkownika, powiązany z modelem wbudowanym \sphinxtitleref{User}.
Zawiera również pole \sphinxtitleref{totp\_device} reprezentujące urządzenie TOTP (Time\sphinxhyphen{}based One\sphinxhyphen{}Time Password).
\begin{quote}\begin{description}
\sphinxlineitem{Zmienne}\begin{itemize}
\item {} 
\sphinxAtStartPar
\sphinxstyleliteralstrong{\sphinxupquote{user}} (\sphinxstyleliteralemphasis{\sphinxupquote{User}}) \textendash{} Powiązanie z modelem użytkownika Django.

\item {} 
\sphinxAtStartPar
\sphinxstyleliteralstrong{\sphinxupquote{totp\_device}} (\sphinxstyleliteralemphasis{\sphinxupquote{TOTPDevice}}) \textendash{} Powiązanie z urządzeniem TOTP (opcjonalne).

\end{itemize}

\end{description}\end{quote}
\index{UserProfile.DoesNotExist@\spxentry{UserProfile.DoesNotExist}}

\begin{fulllineitems}
\phantomsection\label{\detokenize{mainApp:mainApp.models.UserProfile.DoesNotExist}}
\pysigstartsignatures
\pysigline{\sphinxbfcode{\sphinxupquote{exception\DUrole{w}{ }}}\sphinxbfcode{\sphinxupquote{DoesNotExist}}}
\pysigstopsignatures
\sphinxAtStartPar
Klasy bazowe: \sphinxcode{\sphinxupquote{ObjectDoesNotExist}}

\end{fulllineitems}

\index{UserProfile.MultipleObjectsReturned@\spxentry{UserProfile.MultipleObjectsReturned}}

\begin{fulllineitems}
\phantomsection\label{\detokenize{mainApp:mainApp.models.UserProfile.MultipleObjectsReturned}}
\pysigstartsignatures
\pysigline{\sphinxbfcode{\sphinxupquote{exception\DUrole{w}{ }}}\sphinxbfcode{\sphinxupquote{MultipleObjectsReturned}}}
\pysigstopsignatures
\sphinxAtStartPar
Klasy bazowe: \sphinxcode{\sphinxupquote{MultipleObjectsReturned}}

\end{fulllineitems}

\index{id (mainApp.models.UserProfile atrybut)@\spxentry{id}\spxextra{mainApp.models.UserProfile atrybut}}

\begin{fulllineitems}
\phantomsection\label{\detokenize{mainApp:mainApp.models.UserProfile.id}}
\pysigstartsignatures
\pysigline{\sphinxbfcode{\sphinxupquote{id}}}
\pysigstopsignatures
\sphinxAtStartPar
A wrapper for a deferred\sphinxhyphen{}loading field. When the value is read from this
object the first time, the query is executed.

\end{fulllineitems}

\index{objects (mainApp.models.UserProfile atrybut)@\spxentry{objects}\spxextra{mainApp.models.UserProfile atrybut}}

\begin{fulllineitems}
\phantomsection\label{\detokenize{mainApp:mainApp.models.UserProfile.objects}}
\pysigstartsignatures
\pysigline{\sphinxbfcode{\sphinxupquote{objects}}\sphinxbfcode{\sphinxupquote{\DUrole{w}{ }\DUrole{p}{=}\DUrole{w}{ }\textless{}django.db.models.manager.Manager object\textgreater{}}}}
\pysigstopsignatures
\end{fulllineitems}

\index{totp\_device (mainApp.models.UserProfile atrybut)@\spxentry{totp\_device}\spxextra{mainApp.models.UserProfile atrybut}}

\begin{fulllineitems}
\phantomsection\label{\detokenize{mainApp:mainApp.models.UserProfile.totp_device}}
\pysigstartsignatures
\pysigline{\sphinxbfcode{\sphinxupquote{totp\_device}}}
\pysigstopsignatures
\sphinxAtStartPar
Accessor to the related object on the forward side of a one\sphinxhyphen{}to\sphinxhyphen{}one relation.

\sphinxAtStartPar
In the example:

\begin{sphinxVerbatim}[commandchars=\\\{\}]
\PYG{k}{class} \PYG{n+nc}{Restaurant}\PYG{p}{(}\PYG{n}{Model}\PYG{p}{)}\PYG{p}{:}
    \PYG{n}{place} \PYG{o}{=} \PYG{n}{OneToOneField}\PYG{p}{(}\PYG{n}{Place}\PYG{p}{,} \PYG{n}{related\PYGZus{}name}\PYG{o}{=}\PYG{l+s+s1}{\PYGZsq{}}\PYG{l+s+s1}{restaurant}\PYG{l+s+s1}{\PYGZsq{}}\PYG{p}{)}
\end{sphinxVerbatim}

\sphinxAtStartPar
\sphinxcode{\sphinxupquote{Restaurant.place}} is a \sphinxcode{\sphinxupquote{ForwardOneToOneDescriptor}} instance.

\end{fulllineitems}

\index{totp\_device\_id (mainApp.models.UserProfile atrybut)@\spxentry{totp\_device\_id}\spxextra{mainApp.models.UserProfile atrybut}}

\begin{fulllineitems}
\phantomsection\label{\detokenize{mainApp:mainApp.models.UserProfile.totp_device_id}}
\pysigstartsignatures
\pysigline{\sphinxbfcode{\sphinxupquote{totp\_device\_id}}}
\pysigstopsignatures
\end{fulllineitems}

\index{user (mainApp.models.UserProfile atrybut)@\spxentry{user}\spxextra{mainApp.models.UserProfile atrybut}}

\begin{fulllineitems}
\phantomsection\label{\detokenize{mainApp:mainApp.models.UserProfile.user}}
\pysigstartsignatures
\pysigline{\sphinxbfcode{\sphinxupquote{user}}}
\pysigstopsignatures
\sphinxAtStartPar
Accessor to the related object on the forward side of a one\sphinxhyphen{}to\sphinxhyphen{}one relation.

\sphinxAtStartPar
In the example:

\begin{sphinxVerbatim}[commandchars=\\\{\}]
\PYG{k}{class} \PYG{n+nc}{Restaurant}\PYG{p}{(}\PYG{n}{Model}\PYG{p}{)}\PYG{p}{:}
    \PYG{n}{place} \PYG{o}{=} \PYG{n}{OneToOneField}\PYG{p}{(}\PYG{n}{Place}\PYG{p}{,} \PYG{n}{related\PYGZus{}name}\PYG{o}{=}\PYG{l+s+s1}{\PYGZsq{}}\PYG{l+s+s1}{restaurant}\PYG{l+s+s1}{\PYGZsq{}}\PYG{p}{)}
\end{sphinxVerbatim}

\sphinxAtStartPar
\sphinxcode{\sphinxupquote{Restaurant.place}} is a \sphinxcode{\sphinxupquote{ForwardOneToOneDescriptor}} instance.

\end{fulllineitems}

\index{user\_id (mainApp.models.UserProfile atrybut)@\spxentry{user\_id}\spxextra{mainApp.models.UserProfile atrybut}}

\begin{fulllineitems}
\phantomsection\label{\detokenize{mainApp:mainApp.models.UserProfile.user_id}}
\pysigstartsignatures
\pysigline{\sphinxbfcode{\sphinxupquote{user\_id}}}
\pysigstopsignatures
\end{fulllineitems}


\end{fulllineitems}



\subsection{mainApp.serializers module}
\label{\detokenize{mainApp:module-mainApp.serializers}}\label{\detokenize{mainApp:mainapp-serializers-module}}\index{module@\spxentry{module}!mainApp.serializers@\spxentry{mainApp.serializers}}\index{mainApp.serializers@\spxentry{mainApp.serializers}!module@\spxentry{module}}\index{CreditSerializer (klasa w module mainApp.serializers)@\spxentry{CreditSerializer}\spxextra{klasa w module mainApp.serializers}}

\begin{fulllineitems}
\phantomsection\label{\detokenize{mainApp:mainApp.serializers.CreditSerializer}}
\pysigstartsignatures
\pysiglinewithargsret{\sphinxbfcode{\sphinxupquote{class\DUrole{w}{ }}}\sphinxcode{\sphinxupquote{mainApp.serializers.}}\sphinxbfcode{\sphinxupquote{CreditSerializer}}}{\sphinxparam{\DUrole{o}{*}\DUrole{n}{args}}\sphinxparamcomma \sphinxparam{\DUrole{o}{**}\DUrole{n}{kwargs}}}{}
\pysigstopsignatures
\sphinxAtStartPar
Klasy bazowe: \sphinxcode{\sphinxupquote{ModelSerializer}}

\sphinxAtStartPar
Serializer dla modelu \sphinxtitleref{CreditStorage}.

\sphinxAtStartPar
Klasa \sphinxtitleref{CreditSerializer} służy do serializacji modelu \sphinxtitleref{CreditStorage}.
Konwertuje dane modelu na format JSON do wykorzystania w interfejsach API.
\index{CreditSerializer.Meta (klasa w module mainApp.serializers)@\spxentry{CreditSerializer.Meta}\spxextra{klasa w module mainApp.serializers}}

\begin{fulllineitems}
\phantomsection\label{\detokenize{mainApp:mainApp.serializers.CreditSerializer.Meta}}
\pysigstartsignatures
\pysigline{\sphinxbfcode{\sphinxupquote{class\DUrole{w}{ }}}\sphinxbfcode{\sphinxupquote{Meta}}}
\pysigstopsignatures
\sphinxAtStartPar
Klasy bazowe: \sphinxcode{\sphinxupquote{object}}
\index{fields (mainApp.serializers.CreditSerializer.Meta atrybut)@\spxentry{fields}\spxextra{mainApp.serializers.CreditSerializer.Meta atrybut}}

\begin{fulllineitems}
\phantomsection\label{\detokenize{mainApp:mainApp.serializers.CreditSerializer.Meta.fields}}
\pysigstartsignatures
\pysigline{\sphinxbfcode{\sphinxupquote{fields}}\sphinxbfcode{\sphinxupquote{\DUrole{w}{ }\DUrole{p}{=}\DUrole{w}{ }{[}\textquotesingle{}pk\textquotesingle{}, \textquotesingle{}name\textquotesingle{}, \textquotesingle{}icon\textquotesingle{}, \textquotesingle{}username\textquotesingle{}, \textquotesingle{}password\textquotesingle{}{]}}}}
\pysigstopsignatures
\end{fulllineitems}

\index{model (mainApp.serializers.CreditSerializer.Meta atrybut)@\spxentry{model}\spxextra{mainApp.serializers.CreditSerializer.Meta atrybut}}

\begin{fulllineitems}
\phantomsection\label{\detokenize{mainApp:mainApp.serializers.CreditSerializer.Meta.model}}
\pysigstartsignatures
\pysigline{\sphinxbfcode{\sphinxupquote{model}}}
\pysigstopsignatures
\sphinxAtStartPar
alias of {\hyperref[\detokenize{mainApp:mainApp.models.CreditStorage}]{\sphinxcrossref{\sphinxcode{\sphinxupquote{CreditStorage}}}}}

\end{fulllineitems}


\end{fulllineitems}


\end{fulllineitems}



\subsection{mainApp.tests module}
\label{\detokenize{mainApp:module-mainApp.tests}}\label{\detokenize{mainApp:mainapp-tests-module}}\index{module@\spxentry{module}!mainApp.tests@\spxentry{mainApp.tests}}\index{mainApp.tests@\spxentry{mainApp.tests}!module@\spxentry{module}}

\subsection{mainApp.urls module}
\label{\detokenize{mainApp:module-mainApp.urls}}\label{\detokenize{mainApp:mainapp-urls-module}}\index{module@\spxentry{module}!mainApp.urls@\spxentry{mainApp.urls}}\index{mainApp.urls@\spxentry{mainApp.urls}!module@\spxentry{module}}

\subsection{mainApp.views module}
\label{\detokenize{mainApp:module-mainApp.views}}\label{\detokenize{mainApp:mainapp-views-module}}\index{module@\spxentry{module}!mainApp.views@\spxentry{mainApp.views}}\index{mainApp.views@\spxentry{mainApp.views}!module@\spxentry{module}}\index{add\_service() (w module mainApp.views)@\spxentry{add\_service()}\spxextra{w module mainApp.views}}

\begin{fulllineitems}
\phantomsection\label{\detokenize{mainApp:mainApp.views.add_service}}
\pysigstartsignatures
\pysiglinewithargsret{\sphinxcode{\sphinxupquote{mainApp.views.}}\sphinxbfcode{\sphinxupquote{add\_service}}}{\sphinxparam{\DUrole{n}{request}\DUrole{p}{:}\DUrole{w}{ }\DUrole{n}{HttpRequest}}}{}
\pysigstopsignatures
\sphinxAtStartPar
Widok dodawania nowej usługi.

\sphinxAtStartPar
Dodaje nową usługę do bazy danych, wymaga uwierzytelnienia.
Otrzymuje dane usługi (nazwę, użytkownika, hasło, ikonę) i zapisuje je do bazy danych.
\begin{quote}\begin{description}
\sphinxlineitem{Parametry}
\sphinxAtStartPar
\sphinxstyleliteralstrong{\sphinxupquote{request}} \textendash{} HttpRequest

\sphinxlineitem{Zwraca}
\sphinxAtStartPar
JsonResponse z komunikatem o sukcesie lub błędzie

\end{description}\end{quote}

\end{fulllineitems}

\index{authenticate\_view() (w module mainApp.views)@\spxentry{authenticate\_view()}\spxextra{w module mainApp.views}}

\begin{fulllineitems}
\phantomsection\label{\detokenize{mainApp:mainApp.views.authenticate_view}}
\pysigstartsignatures
\pysiglinewithargsret{\sphinxcode{\sphinxupquote{mainApp.views.}}\sphinxbfcode{\sphinxupquote{authenticate\_view}}}{\sphinxparam{\DUrole{n}{request}}}{}
\pysigstopsignatures
\sphinxAtStartPar
Widok uwierzytelniania użytkownika z dwuetapową weryfikacją.

\sphinxAtStartPar
Obsługuje żądania typu POST zawierające dane uwierzytelniania i kod dwuetapowy TOTP.
Dokonuje uwierzytelniania użytkownika i kodu TOTP, zwraca komunikat o sukcesie lub błędzie.
\begin{quote}\begin{description}
\sphinxlineitem{Parametry}
\sphinxAtStartPar
\sphinxstyleliteralstrong{\sphinxupquote{request}} \textendash{} HttpRequest

\sphinxlineitem{Zwraca}
\sphinxAtStartPar
JsonResponse z komunikatem o sukcesie lub błędzie

\end{description}\end{quote}

\end{fulllineitems}

\index{delete\_service() (w module mainApp.views)@\spxentry{delete\_service()}\spxextra{w module mainApp.views}}

\begin{fulllineitems}
\phantomsection\label{\detokenize{mainApp:mainApp.views.delete_service}}
\pysigstartsignatures
\pysiglinewithargsret{\sphinxcode{\sphinxupquote{mainApp.views.}}\sphinxbfcode{\sphinxupquote{delete\_service}}}{\sphinxparam{\DUrole{n}{request}\DUrole{p}{:}\DUrole{w}{ }\DUrole{n}{HttpRequest}}\sphinxparamcomma \sphinxparam{\DUrole{n}{id}\DUrole{p}{:}\DUrole{w}{ }\DUrole{n}{int}}}{}
\pysigstopsignatures
\sphinxAtStartPar
Widok usuwania usługi.

\sphinxAtStartPar
Usuwa usługę z bazy danych na podstawie przekazanego identyfikatora.
Wymaga uwierzytelnienia.
\begin{quote}\begin{description}
\sphinxlineitem{Parametry}\begin{itemize}
\item {} 
\sphinxAtStartPar
\sphinxstyleliteralstrong{\sphinxupquote{request}} \textendash{} HttpRequest

\item {} 
\sphinxAtStartPar
\sphinxstyleliteralstrong{\sphinxupquote{id}} \textendash{} Identyfikator usługi do usunięcia

\end{itemize}

\sphinxlineitem{Zwraca}
\sphinxAtStartPar
JsonResponse z komunikatem o sukcesie lub błędzie

\end{description}\end{quote}

\end{fulllineitems}

\index{index() (w module mainApp.views)@\spxentry{index()}\spxextra{w module mainApp.views}}

\begin{fulllineitems}
\phantomsection\label{\detokenize{mainApp:mainApp.views.index}}
\pysigstartsignatures
\pysiglinewithargsret{\sphinxcode{\sphinxupquote{mainApp.views.}}\sphinxbfcode{\sphinxupquote{index}}}{\sphinxparam{\DUrole{n}{request}}}{}
\pysigstopsignatures
\sphinxAtStartPar
Widok wyświetlający prosty komunikat „Hello there from index!”.
\begin{quote}\begin{description}
\sphinxlineitem{Parametry}
\sphinxAtStartPar
\sphinxstyleliteralstrong{\sphinxupquote{request}} \textendash{} HttpRequest

\sphinxlineitem{Zwraca}
\sphinxAtStartPar
HttpResponse z komunikatem

\end{description}\end{quote}

\end{fulllineitems}

\index{list\_services() (w module mainApp.views)@\spxentry{list\_services()}\spxextra{w module mainApp.views}}

\begin{fulllineitems}
\phantomsection\label{\detokenize{mainApp:mainApp.views.list_services}}
\pysigstartsignatures
\pysiglinewithargsret{\sphinxcode{\sphinxupquote{mainApp.views.}}\sphinxbfcode{\sphinxupquote{list\_services}}}{\sphinxparam{\DUrole{n}{request}\DUrole{p}{:}\DUrole{w}{ }\DUrole{n}{HttpRequest}}}{}
\pysigstopsignatures
\sphinxAtStartPar
Widok listowania usług użytkownika.

\sphinxAtStartPar
Zwraca listę usług użytkownika z bazy danych.
Wymaga uwierzytelnienia.
\begin{quote}\begin{description}
\sphinxlineitem{Parametry}
\sphinxAtStartPar
\sphinxstyleliteralstrong{\sphinxupquote{request}} \textendash{} HttpRequest

\sphinxlineitem{Zwraca}
\sphinxAtStartPar
JsonResponse z listą usług lub komunikatem o błędzie

\end{description}\end{quote}

\end{fulllineitems}

\index{login\_view() (w module mainApp.views)@\spxentry{login\_view()}\spxextra{w module mainApp.views}}

\begin{fulllineitems}
\phantomsection\label{\detokenize{mainApp:mainApp.views.login_view}}
\pysigstartsignatures
\pysiglinewithargsret{\sphinxcode{\sphinxupquote{mainApp.views.}}\sphinxbfcode{\sphinxupquote{login\_view}}}{\sphinxparam{\DUrole{n}{request}}}{}
\pysigstopsignatures
\sphinxAtStartPar
Widok logowania użytkownika.

\sphinxAtStartPar
Obsługuje żądania typu POST zawierające dane logowania.
Zwraca komunikat po pomyślnym zalogowaniu lub błąd przy nieudanym logowaniu.
\begin{quote}\begin{description}
\sphinxlineitem{Parametry}
\sphinxAtStartPar
\sphinxstyleliteralstrong{\sphinxupquote{request}} \textendash{} HttpRequest

\sphinxlineitem{Zwraca}
\sphinxAtStartPar
JsonResponse z komunikatem o sukcesie lub błędzie

\end{description}\end{quote}

\end{fulllineitems}

\index{register\_view() (w module mainApp.views)@\spxentry{register\_view()}\spxextra{w module mainApp.views}}

\begin{fulllineitems}
\phantomsection\label{\detokenize{mainApp:mainApp.views.register_view}}
\pysigstartsignatures
\pysiglinewithargsret{\sphinxcode{\sphinxupquote{mainApp.views.}}\sphinxbfcode{\sphinxupquote{register\_view}}}{\sphinxparam{\DUrole{n}{request}}}{}
\pysigstopsignatures
\sphinxAtStartPar
Widok rejestracji nowego użytkownika.

\sphinxAtStartPar
Obsługuje żądania typu POST zawierające dane niezbędne do rejestracji nowego użytkownika.
Po sukcesie zwraca komunikat o rejestracji oraz wygenerowany sekret TOTP.
\begin{quote}\begin{description}
\sphinxlineitem{Parametry}
\sphinxAtStartPar
\sphinxstyleliteralstrong{\sphinxupquote{request}} \textendash{} HttpRequest

\sphinxlineitem{Zwraca}
\sphinxAtStartPar
JsonResponse z komunikatem o sukcesie lub błędzie

\end{description}\end{quote}

\end{fulllineitems}

\index{status() (w module mainApp.views)@\spxentry{status()}\spxextra{w module mainApp.views}}

\begin{fulllineitems}
\phantomsection\label{\detokenize{mainApp:mainApp.views.status}}
\pysigstartsignatures
\pysiglinewithargsret{\sphinxcode{\sphinxupquote{mainApp.views.}}\sphinxbfcode{\sphinxupquote{status}}}{\sphinxparam{\DUrole{n}{request}}}{}
\pysigstopsignatures
\sphinxAtStartPar
Widok zwracający status uwierzytelnienia użytkownika.
\begin{quote}\begin{description}
\sphinxlineitem{Parametry}
\sphinxAtStartPar
\sphinxstyleliteralstrong{\sphinxupquote{request}} \textendash{} HttpRequest

\sphinxlineitem{Zwraca}
\sphinxAtStartPar
JsonResponse z informacją o stanie uwierzytelnienia

\end{description}\end{quote}

\end{fulllineitems}

\index{user\_logout() (w module mainApp.views)@\spxentry{user\_logout()}\spxextra{w module mainApp.views}}

\begin{fulllineitems}
\phantomsection\label{\detokenize{mainApp:mainApp.views.user_logout}}
\pysigstartsignatures
\pysiglinewithargsret{\sphinxcode{\sphinxupquote{mainApp.views.}}\sphinxbfcode{\sphinxupquote{user\_logout}}}{\sphinxparam{\DUrole{n}{request}}}{}
\pysigstopsignatures
\sphinxAtStartPar
Widok wylogowania użytkownika.

\sphinxAtStartPar
Wylogowuje użytkownika z systemu.
\begin{quote}\begin{description}
\sphinxlineitem{Parametry}
\sphinxAtStartPar
\sphinxstyleliteralstrong{\sphinxupquote{request}} \textendash{} HttpRequest

\sphinxlineitem{Zwraca}
\sphinxAtStartPar
JsonResponse z komunikatem o wylogowaniu

\end{description}\end{quote}

\end{fulllineitems}



\subsection{Module contents}
\label{\detokenize{mainApp:module-mainApp}}\label{\detokenize{mainApp:module-contents}}\index{module@\spxentry{module}!mainApp@\spxentry{mainApp}}\index{mainApp@\spxentry{mainApp}!module@\spxentry{module}}
\sphinxstepscope


\section{manage module}
\label{\detokenize{manage:module-manage}}\label{\detokenize{manage:manage-module}}\label{\detokenize{manage::doc}}\index{module@\spxentry{module}!manage@\spxentry{manage}}\index{manage@\spxentry{manage}!module@\spxentry{module}}
\sphinxAtStartPar
Django’s command\sphinxhyphen{}line utility for administrative tasks.
\index{main() (w module manage)@\spxentry{main()}\spxextra{w module manage}}

\begin{fulllineitems}
\phantomsection\label{\detokenize{manage:manage.main}}
\pysigstartsignatures
\pysiglinewithargsret{\sphinxcode{\sphinxupquote{manage.}}\sphinxbfcode{\sphinxupquote{main}}}{}{}
\pysigstopsignatures
\sphinxAtStartPar
Run administrative tasks.

\end{fulllineitems}



\chapter{Indices and tables}
\label{\detokenize{index:indices-and-tables}}\begin{itemize}
\item {} 
\sphinxAtStartPar
\DUrole{xref,std,std-ref}{genindex}

\item {} 
\sphinxAtStartPar
\DUrole{xref,std,std-ref}{modindex}

\item {} 
\sphinxAtStartPar
\DUrole{xref,std,std-ref}{search}

\end{itemize}


\renewcommand{\indexname}{Indeks modułów Pythona}
\begin{sphinxtheindex}
\let\bigletter\sphinxstyleindexlettergroup
\bigletter{b}
\item\relax\sphinxstyleindexentry{backend}\sphinxstyleindexpageref{backend:\detokenize{module-backend}}
\item\relax\sphinxstyleindexentry{backend.asgi}\sphinxstyleindexpageref{backend:\detokenize{module-backend.asgi}}
\item\relax\sphinxstyleindexentry{backend.settings}\sphinxstyleindexpageref{backend:\detokenize{module-backend.settings}}
\item\relax\sphinxstyleindexentry{backend.urls}\sphinxstyleindexpageref{backend:\detokenize{module-backend.urls}}
\item\relax\sphinxstyleindexentry{backend.wsgi}\sphinxstyleindexpageref{backend:\detokenize{module-backend.wsgi}}
\indexspace
\bigletter{m}
\item\relax\sphinxstyleindexentry{mainApp}\sphinxstyleindexpageref{mainApp:\detokenize{module-mainApp}}
\item\relax\sphinxstyleindexentry{mainApp.admin}\sphinxstyleindexpageref{mainApp:\detokenize{module-mainApp.admin}}
\item\relax\sphinxstyleindexentry{mainApp.apps}\sphinxstyleindexpageref{mainApp:\detokenize{module-mainApp.apps}}
\item\relax\sphinxstyleindexentry{mainApp.forms}\sphinxstyleindexpageref{mainApp:\detokenize{module-mainApp.forms}}
\item\relax\sphinxstyleindexentry{mainApp.models}\sphinxstyleindexpageref{mainApp:\detokenize{module-mainApp.models}}
\item\relax\sphinxstyleindexentry{mainApp.serializers}\sphinxstyleindexpageref{mainApp:\detokenize{module-mainApp.serializers}}
\item\relax\sphinxstyleindexentry{mainApp.tests}\sphinxstyleindexpageref{mainApp:\detokenize{module-mainApp.tests}}
\item\relax\sphinxstyleindexentry{mainApp.urls}\sphinxstyleindexpageref{mainApp:\detokenize{module-mainApp.urls}}
\item\relax\sphinxstyleindexentry{mainApp.views}\sphinxstyleindexpageref{mainApp:\detokenize{module-mainApp.views}}
\item\relax\sphinxstyleindexentry{manage}\sphinxstyleindexpageref{manage:\detokenize{module-manage}}
\end{sphinxtheindex}

\renewcommand{\indexname}{Indeks}
\printindex
\end{document}